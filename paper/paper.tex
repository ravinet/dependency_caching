\documentclass[10pt,twocolumn,letterpaper]{article}
%\documentclass[10pt on 12pt, twocolumn]{article}
\usepackage[usenames, dvipsnames]{xcolor}
\usepackage{usenix, times,color,graphicx,amsmath,array,amssymb,subfigure}
%\usepackage{usenix,times,color,graphicx,url,amsmath,array,amssymb,subfigure
\usepackage{mathptm,pst-all,times,helvet,courier,xspace}
\usepackage{boxedminipage,multirow,endnotes,balance}
\usepackage{pdfpages}
\usepackage{rotating}
\usepackage{colortbl}
\usepackage{transparent}
\usepackage{appendix}
\usepackage[obeyspaces]{url}

\urlstyle{rm}
\usepackage{ifthen}
\usepackage{fancyhdr}
%\usepackage{mathptmx}   % Times + Times-like math symbols
%\usepackage{courier}
%\usepackage[scaled=0.92]{helvet}
%\usepackage{color}
%\usepackage{colortbl}

\pdfoptionpdfminorversion=6

\newfont{\dft}{phvb at 6pt}
\newfont{\mft}{phvro at 6pt}
\newfont{\df}{phvb at 9pt}
\newfont{\mf}{phvro at 9pt}
\frenchspacing

\newpsobject{showgrid}{psgrid}{subgriddiv=2,griddots=10,gridlabels=6pt}
\newcommand{\note}[1]{{\bf [ NOTE: #1 ]}} 
\newcommand{\fixme}[1]{{\bf [ XXX: #1 ]}}

\makeatletter

\global\def\section{\@startsection {section}{1}{\z@}%
                                   {-1.5ex \@plus -0.8ex \@minus -.1ex}%
                                   {0.6ex \@plus.2ex}% changed this
                                   {\normalfont\bfseries\scshape\fontsize{11}{13}\selectfont}}
\global\def\subsection{\@startsection{subsection}{2}{\z@}%
                                     {-1.25ex\@plus -0.8ex \@minus -.1ex}%
                                     {0.3ex \@plus .1ex}% changed this
                                     {\normalfont\bfseries\fontsize{10}{12}\selectfont}}
\global\def\subsubsection{\@startsection{subsubsection}{3}{\z@}%
                                     {-1ex\@plus -1ex \@minus -.1ex}%
                                     {0.1ex \@plus .1ex}% changed this
                                     {\normalfont\itshape\fontsize{10}{12}\selectfont}}


\usepackage{bibspacing}
\setlength{\bibspacing}{\baselineskip}

% peanut gallery comments; from Mike Walfish

%
%
% NOTE: Comment out the line below if you want a draft with no red comments.
% NOTE: Commenting out this line may replace some of the red comments with 
%       extra spaces or newlines.
\def\noeditingmarks{}
%
\newcommand{\textred}[1]{\textcolor{red}{#1}}
\ifx\noeditingmarks\undefined
   \newcommand{\pgwrapper}[2]{\textred{#1: #2}}
\else
   \newcommand{\pgwrapper}[2]{}
\fi
\newcommand{\lenin}[1]{\pgwrapper{Lenin}{#1}}
\newcommand{\calvin}[1]{\pgwrapper{Calvin}{#1}}
\newcommand{\srm}[1]{\pgwrapper{Sam}{#1}}
\newcommand{\todo}[1]{\pgwrapper{TODO}{#1}}
\newcommand{\hb}[1]{\pgwrapper{Hari}{#1}}
\newcommand{\anirudh}[1]{\pgwrapper{anirudh}{#1}}
    \renewcommand{\textfraction}{0.07}	% allow minimal text w. figs

% end peanut gallery comments

\newcommand{\ea}{et al.}
\newcommand{\ie}{i.e.}
\newcommand{\eg}{e.g.}
\newcommand{\Prob}{{\mathbb P}}

%\let\footnote=\endnote

%%% from usenix camera-ready template
\date{}
%\renewcommand{\footnote}{\endnote}
%%%%
\begin{document}

% Compact itemize and enumerate.  Note that they use the same counters and
% symbols as the usual itemize and enumerate environments.
\def\compactify{\itemsep=0pt \topsep=0pt \partopsep=0pt \parsep=0pt}
\let\latexusecounter=\usecounter
\newenvironment{CompactItemize}
  {\def\usecounter{\compactify\latexusecounter}
   \begin{itemize}}
  {\end{itemize}\let\usecounter=\latexusecounter}
\newenvironment{CompactEnumerate}
  {\def\usecounter{\compactify\latexusecounter}
   \begin{enumerate}}
  {\end{enumerate}\let\usecounter=\latexusecounter}

\twocolumn[\begin{@twocolumnfalse}

\begin{centering}
{\large \bf Web Page Dependencies}

\vspace{\baselineskip}

Authors

\textit{MIT, MSR} \\
%\texttt{\{ravinet\}@mit.edu}

\end{centering}

\vspace{\baselineskip}

\end{@twocolumnfalse}]

\pagestyle{plain}
\interfootnotelinepenalty 100000
\widowpenalty 100000
\clubpenalty 100000
\newfont{\tf}{phvro at 9.5pt}
\newfont{\tft}{phvro at 7.25pt}

\begin{sloppypar}

\begin{abstract}
\input{abstract}
\end{abstract}
\section{Introduction}

\section{Related Work}

\section{Design}

\section{Experimental Setup}

\section{Results}

% tables that summarize the graphs before and after our new dependencies
\begin{table*}[tb]
\centering
\small
\begin{tabular}{|c|c|c|}
\hline
 & original & with new dependencies \\
\hline
length of critical path & 4 & 8 \\
number of critical paths & 14 & 5 \\
\% of slack nodes & 79\% & 82\% \\
number of nodes & 82 & 82 \\
number of edges & 82 & 160 \\
\hline
\end{tabular}
\caption{Comparison of weather.com dependency graphs with and without our window and document dependencies.}
\label{t:weathergraph}
\end{table*}

\begin{table*}[tb]
\centering
\small
\begin{tabular}{|c|c|c|}
\hline
 & original & with new dependencies \\
\hline
length of critical path & 2 & 5 \\
number of critical paths & 27 & 1 \\
\% of slack nodes & 0\% & 82\% \\
number of nodes & 28 & 28 \\
number of edges & 27 & 38 \\
\hline
\end{tabular}
\caption{Comparison of apple.com dependency graphs with and without our window and document dependencies.}
\label{t:applegraph}
\end{table*}

\begin{table*}[tb]
\centering
\small
\begin{tabular}{|c|c|c|}
\hline
 & original & with new dependencies \\
\hline
length of critical path & 5 & 7 \\
number of critical paths & 2 & 1 \\
\% of slack nodes & 95\% & 94\% \\
number of nodes & 117 & 117 \\
number of edges & 117 & 149 \\
\hline
\end{tabular}\caption{Comparison of espn.go.com dependency graphs with and without our window and document dependencies.}
\label{t:espngraph}
\end{table*}

\begin{table*}[tb]
\centering
\small
\begin{tabular}{|c|c|c|}
\hline
 & original & with new dependencies \\
\hline
length of critical path & 4 & 5 \\
number of critical paths & 2 & 1 \\
\% of slack nodes & 81\% & 86\% \\
number of nodes & 37 & 37 \\
number of edges & 36 & 54 \\
\hline
\end{tabular}\caption{Comparison of stackoverflow.com dependency graphs with and without our window and document dependencies.}
\label{t:stackoverflowgraph}
\end{table*}

\begin{table*}[tb]
\centering
\small
\begin{tabular}{|c|c|c|}
\hline
 & original & with new dependencies \\
\hline
length of critical path & 4 & 7 \\
number of critical paths & 5 & 3 \\
\% of slack nodes & 88\% & 91\% \\
number of nodes & 102 & 102 \\
number of edges & 101 & 132 \\
\hline
\end{tabular}\caption{Comparison of imgur.com dependency graphs with and without our window and document dependencies.}
\label{t:imgurgraph}
\end{table*}

\begin{table*}[tb]
\centering
\small
\begin{tabular}{|c|c|c|}
\hline
 & original & with new dependencies \\
\hline
length of critical path & 4 & 5 \\
number of critical paths & 1 & 1 \\
\% of slack nodes & 87\% & 83\% \\
number of nodes & 30 & 30 \\
number of edges & 29 & 55 \\
\hline
\end{tabular}\caption{Comparison of bing.com dependency graphs with and without our window and document dependencies.}
\label{t:binggraph}
\end{table*}

\begin{table*}[tb]
\centering
\small
\begin{tabular}{|c|c|c|}
\hline
 & original & with new dependencies \\
\hline
length of critical path & 4 & 4 \\
number of critical paths & 1 & 1 \\
\% of slack nodes & 91\% & 91\% \\
number of nodes & 46 & 46 \\
number of edges & 45 & 46 \\
\hline
\end{tabular}\caption{Comparison of ebay.com dependency graphs with and without our window and document dependencies.}
\label{t:ebaygraph}
\end{table*}

\begin{table*}[tb]
\centering
\small
\begin{tabular}{|c|c|c|}
\hline
 & original & with new dependencies \\
\hline
length of critical path & 6 & 18 \\
number of critical paths & 1 & 2 \\
\% of slack nodes & 94\% & 81\% \\
number of nodes & 111 & 111 \\
number of edges & 114 & 317 \\
\hline
\end{tabular}\caption{Comparison of m.finishline.com dependency graphs with and without our window and document dependencies.}
\label{t:finishlinegraph}
\end{table*}


\begin{table*}[tb]
\centering
\small
\begin{tabular}{|c|c|c|}
\hline
 & original & with new dependencies \\
\hline
length of critical path & 7 & 12 \\
number of critical paths & 4 & 1 \\
\% of slack nodes & 80\% & 82\% \\
number of nodes & 65 & 65 \\
number of edges & 72 & 110 \\
\hline
\end{tabular}\caption{Comparison of ask.com dependency graphs with and without our window and document dependencies.}
\label{t:askgraph}
\end{table*}

\begin{table*}[tb]
\centering
\small
\begin{tabular}{|c|c|c|}
\hline
 & original & with new dependencies \\
\hline
length of critical path & 5 & 17 \\
number of critical paths & 3 & 3 \\
\% of slack nodes & 90\% & 83\% \\
number of nodes & 107 & 112 \\
number of edges & 108 & 276 \\
\hline
\end{tabular}\caption{Comparison of cbssports.com dependency graphs with and without our window and document dependencies.}
\label{t:cbssportsgraph}
\end{table*}

\begin{table*}[tb]
\centering
\small
\begin{tabular}{|c|c|c|}
\hline
 & original & with new dependencies \\
\hline
length of critical path & 4 & 9 \\
number of critical paths & 12 & 1 \\
\% of slack nodes & 80\% & 89\% \\
number of nodes & 81 & 81 \\
number of edges & 82 & 148 \\
\hline
\end{tabular}\caption{Comparison of outbrain.com dependency graphs with and without our window and document dependencies.}
\label{t:outbraingraph}
\end{table*}


\begin{table*}[tb]
\centering
\small
\begin{tabular}{|c|c|c|}
\hline
 & original & with new dependencies \\
\hline
length of critical path & 4 & 9 \\
number of critical paths & 3 & 1 \\
\% of slack nodes & 86\% & 84\% \\
number of nodes & 58 & 58 \\
number of edges & 58 & 92 \\
\hline
\end{tabular}\caption{Comparison of about.com dependency graphs with and without our window and document dependencies.}
\label{t:aboutgraph}
\end{table*}


\begin{table*}[tb]
\centering
\small
\begin{tabular}{|c|c|c|}
\hline
 & original & with new dependencies \\
\hline
length of critical path & 4 & 5 \\
number of critical paths & 3 & 2 \\
\% of slack nodes & 76\% & 80\% \\
number of nodes & 41 & 41 \\
number of edges & 40 & 55 \\
\hline
\end{tabular}\caption{Comparison of mobile.fandango.com dependency graphs with and without our window and document dependencies.}
\label{t:fandangograph}
\end{table*}

\section{Discussion}

\section{Limitations and Future Work}

\section{Conclusion}
\cite{wprof}

\balance
\end{sloppypar}


%\vfill
{%\footnotesize
%\begin{flushleft}


\bibliographystyle{abbrv}
\bibliography{paper}


%\end{flushleft}
}

%% \begingroup
%% \def\enotesize{\footnotesize}
%% \theendnotes
%% \endgroup
%\theendnotes
\end{document}

